%
% glossary.tex
%
% Fail lühendite/akronüümide ja mõistete defineerimiseks.
% Tekstis kasutatud mõisted ilmuvad automaatselt eraldi leheküljele 
% mõistete selgitusse.
%
% Kasutamine:
%   \gls{} - väikese algustähega mõiste/lühend. Lühendi puhul ilmub esimesel
%            korral teksti:
%               tähendus (LHND)
%            Edasisel kasutamisel tekitab see ainult lühendi.
%   \Gls{} - samasugune nagu \gls{}, suure algustähega
%   \glspl{}, \Glspl{} - mitmuse vormid eelnevatest
%

% Akronüüm koos nii lühendi kui ka selle tähenduse mitmustega
\newacronym[\glsshortpluralkey={KTA-d},longplural=kolmetähelised akronüümid]{kta}{KTA}{kolmetäheline akronüüm}

% Lihtsalt akronüüm ilma mitmusteta
\newacronym{aim}{AIM}{akronüüm ilma mitmuseta}

% Tavaline mõiste
\newglossaryentry{tavaline}
{
  name=tavaline mõiste,
  description={tavalise mõiste veel tavalisem seletus}
}